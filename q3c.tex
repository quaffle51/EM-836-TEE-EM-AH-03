% !TeX root = ./TMA03.tex
From \bth{7.3}{2}{27} if $r<m$ then $RM(r,m)^\perp = RM(m-r-1,m)$.

\begin{table}[h!]
\centering
\begin{tabular}{@{}lcccc|cl@{}}
\toprule
$r$&$m$ & $m-2$ & $RM(r,m)^\perp$ & $RM(m-2,m)^\perp$ &$RM(m-r-1,m)$ &  \\ \midrule
1 & $3$ & $1$ & $RM(1,3)$ & $RM(1,3)$  &$RM(1,3)$&  \\
2 & $4$ & $2$ & $RM(2,4)$ & $RM(2,4)$  &$RM(1,4)$&  \\ 
3 & $5$ & $3$ & $RM(3,5)$ & $RM(3,5)$  &$RM(1,5)$&  \\ 
4 & $6$ & $4$ & $RM(4,6)$ & $RM(4,6)$  &$RM(1,6)$&  \\ 
\vdots &\vdots& \vdots & \vdots &\vdots &\vdots &  \\ 
 \bottomrule
\end{tabular}
\caption{The dual code of $RM(m-2,m)$ is $RM(1,m)$ for $m>2$.}
\label{tab:rmduals}
\end{table}
From Table~\ref{tab:rmduals} we can see that the dual of the  $RM(1,m)$ code is the $RM(m-2,m)$ code for $m>2$. From \hth{13.6}{168} we have the MacWilliams identity we obtain
\begin{equation}
\label{eq:mac}
W_{C^\perp}(z) = \frac{1}{q^{k}}\left[1+(q-1)z\right]^n W_{C}\left(\frac{1-z}{1+(q-1)z}\right)
\end{equation}
Now, from \bth{7.1}{2}{25}, $RM(r,m)$ is a binary linear $[n,k]$-code where $n=2^m$ and $k=\sum_{i=0}^r\binom{m}{i}$. From Table~\ref{tab:rmduals} it is seen that in the right most column that $r=1$ in all the rows. Thus,
\[
	k = \sum_{i=0}^1\binom{m}{i} = \binom{m}{0} + \binom{m}{1} = 1 + m.
\]
In \ref{eq:mac} $q=2$ and $k=m+1$ so \ref{eq:mac} becomes
\begin{equation}
\label{eq:macbecomes}
W_{C^\perp}(z) = \frac{1}{2^{k}}\left[1+z\right]^n W_{C}\left(\frac{1-z}{1+z}\right).
\end{equation}
From part~(b) we have the weight enumerator for $RM(1,m)$, $m\geq1$ 
\[
	W_C(z) = 1 + (2^{m+1} - 2)z^{2^{m-1}} + z^{2m}.
\]
Thus,
\[
	W_C\left(\frac{1-z}{1+z}\right) = 1 + (2^{m+1} - 2)\left(\frac{1-z}{1+z}\right)^{2^{m-1}} + \left(\frac{1-z}{1+z}\right)^{2m}.
\]
Substituting this last expression into the \rhs of \eqref{eq:macbecomes} we obtain
\begin{align*}
W_{C^\perp}(z) &= \frac{1}{2^{k}}\left[1+z\right]^n \left(1 + (2^{m+1} - 2)\left(\frac{1-z}{1+z}\right)^{2^{m-1}} + \left(\frac{1-z}{1+z}\right)^{2^m}\right),\\
&=\frac{1}{2^{k}} \left(\left(1+z\right)^n + \left(1+z\right)^n(2^{m+1} - 2)\left(\frac{1-z}{1+z}\right)^{2^{m-1}} + \left(1+z\right)^n\left(\frac{1-z}{1+z}\right)^{2^m}\right),\\
\end{align*}
It was established above that $n=2^m$ and $k=m+1$, thus we obtain
\begin{align*}
W_{C^\perp}(z)
&=\frac{1}{2^{m+1}} \left(\left(1+z\right)^{2^m} + \left(1+z\right)^{2^m}(2^{m+1} - 2)\left(\frac{1-z}{1+z}\right)^{2^{m-1}} + \left(1+z\right)^n\left(\frac{1-z}{1+z}\right)^{2^m}\right),\\
&=\frac{1}{2^{m+1}} \left(\left(1+z\right)^{2^m} + (2^{m+1} - 2)(1-z^2)^{2^{m-1}} + (1-z)^{2^m}\right), \text{ as required.}
\end{align*}
