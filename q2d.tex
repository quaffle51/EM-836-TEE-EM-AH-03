% !TeX root = ./TMA03.tex
\begin{octavecode}
function f = binom(n, k);
  f = int64(factorial(n)/(factorial(n-k)*factorial(k)));
endfunction
\end{octavecode}
\begin{octavecode}
n = 90;
d = 5;
k = 73;
q = 2;
\end{octavecode}
To prove that there is a binary linear $(90, 2^{73},5)$-code, that is a $[90, 73, 5]$-code, use will be made of \hth{8.10}{91} as follows.

From this theorem which assumes that $q$ is a prime power, then there exists a $q$-ary $[n,k]$-code with minimum distance at least $d$ provided the following inequality holds:
\begin{align*}
	\sum_{i=0}^{d-2}(q-1)^{i}\binom{n-1}{i} < q^{n-k}.
\end{align*}
For the code under consideration $q=\octavec{disp(q)}$ (a prime power), $n=\octavec{disp(n)}$, $k=\octavec{disp(k)}$ and $d=\octavec{disp(d)}$, therefore
\begin{align*}
	\sum_{i=0}^{\octavec{disp(d)}-2}(\octavec{disp(q)}-1)^{i}\binom{\octavec{disp(n)}-1}{i} &< 2^{\octavec{disp(n)}-\octavec{disp(k)}},\\
	\sum_{i=0}^{\octavec{disp(d-2)}}(1)^{i}\binom{\octavec{disp(n-1)}}{i} &< 2^{\octavec{disp(n-k)}},\\
	\octavec{disp(binom(n-1,0))} + \octavec{disp(binom(n-1,1))} + \octavec{disp(binom(n-1,2))} + \octavec{disp(binom(n-1,3))} = 
	\octavec{disp( binom(n-1,0) + 
	               binom(n-1,1) + 
	               binom(n-1,2) + 
	               binom(n-1,3)
	              )
	        } &< \octavec{disp(2^(n-k))} = 2^{\octavec{disp(n-k)}},
\end{align*}
showing that there is a binary linear $(90, 2^{73}, 5)$-code.

